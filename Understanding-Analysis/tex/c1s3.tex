\documentclass[11pt]{article}

\usepackage{graphicx}
\usepackage{amsmath,amssymb}
\usepackage{tikz}
\usepackage{textcomp}

\newcommand{\doctitle}{ }
\makeatletter
\renewcommand{\ps@headings}{%
\renewcommand{\@evenhead}{\parbox{\textwidth}{\hrulefill
\fbox{\sffamily Textbook - Author }\hrulefill } }%
\renewcommand{\@oddhead}{\parbox{\textwidth}{\hrulefill
\fbox{\sffamily Textbook - Author  }\hrulefill}}}
\makeatother


\linespread{.9}
\hoffset=-0in    \voffset=-.5in
\oddsidemargin=0in   \evensidemargin=0in
\topmargin=-.25in
\textwidth=6.5in   \textheight=9.5in
\columnseprule=.3pt
\setlength{\parskip}{1em}



\usepackage{multirow}
\usepackage{multicol}

\newcommand{\mytitle}[1]{{
\begin{centering}
{\Large \sffamily Textbook - Author }\\
\bigskip\bigskip{\Large \sffamily \bfseries{#1}}\\
\bigskip\bigskip\end{centering}
}}

\newcommand{\mytitlecompact}[1]{{

\hfill
{\Large \sffamily \bfseries{#1}}
\hfill
%\bigskip
}}


\newcommand{\mysection}[1]{{
\smallskip
\noindent
{\large \sffamily \bfseries{#1}}
}}


\newcommand{\mysubsection}[1]{{
\smallskip
{\sffamily \bfseries{#1}}
}}

\pagestyle{empty}
\pagestyle{headings}

\def\N{{\mathbb{N}}}
\def\Z{{\mathbb{Z}}}
\def\Q{{\mathbb{Q}}}
\def\R{{\mathbb{R}}}
\def\C{{\mathbb{C}}}

\usepackage{amsthm}
\newtheorem{innercustomthm}{Theorem}
\newenvironment{thm}[1]
    {\renewcommand\theinnercustomthm{#1}\innercustomthm}
    {\endinnercustomthm}

\newtheorem{innercustomdefn}{Definition}
\newenvironment{defn}[1]
    {\renewcommand\theinnercustomdefn{#1}\innercustomdefn\upshape}
    {\endinnercustomdefn}

\newtheorem{innercustomexmp}{Example}
\newenvironment{exmp}[1]
    {\renewcommand\theinnercustomexmp{#1}\innercustomexmp\upshape}
    {\endinnercustomexmp}

\newtheorem{innercustomexer}{Exercise}
\newenvironment{exer}[1]
    {\renewcommand\theinnercustomexer{#1}\innercustomexer\upshape}
    {\endinnercustomexer}

\newtheorem{innercustomlemma}{Lemma}
\newenvironment{lemma}[1]
    {\renewcommand\theinnercustomlemma{#1}\innercustomlemma}
    {\endinnercustomlemma}


\begin{document}

\mytitlecompact{Section 1.3 - Results}

\begin{defn}{1.3.1}
    A set $A\subseteq\R$ is \underline{bounded above} if there exists a number $b\in\R$ such that $a\leq b$ for all $a\in A$. The number $b$ is called an \underline{upper bound} for $A$. Similarly, the set $A$ is \underline{bounded below} if there exists a \underline{lower bound} $\l\in\R$ satisfying $\l\leq a$ for every $a\in A$.
\end{defn}

\begin{defn}{1.3.2}
    A real number $s$ is the \underline{least upper bound} or \underline{supremum} for a set $A\in\R$ if it meets the following criteria:
    \begin{enumerate}
        \item[(i)]
            $s$ is an upper bound for $A$
        \item[(ii)]
            if $b$ is any upper bound for $A$, then $s\leq b$.
    \end{enumerate}
\end{defn}

\begin{defn}{1.3.4}
    A real number $a_0$ is a \underline{maximum} of the set $A$ if $a_0$ is an element of $A$ and $a_0\geq a$ for all $a\in A$. Similarly, a number $a_1$ is a \underline{minimum} of $A$ if $a_1\leq a$ for every $a\in A$.
\end{defn}

\begin{lemma}{1.3.8}
    Assume $s\in\R$ is an upper bound for a set $A\subseteq\R$. Then, $s=\mathrm{sup}A$ if and only if, for every choice of $\epsilon>0$, there exists an element $a\in A$ satisfying $s-\epsilon<a$.
\end{lemma}
\begin{proof}
    For the forward direction we assume that $s=\sup A$ and we begin by observing that $s-\epsilon<s$ for all $\epsilon>0$. Since $s$ is the supremum for $A$ it must be the case that $s-\epsilon$ is not an upper bound. Therefore, there exists $a\in A$ for our arbitrary choice of $\epsilon$ satisfying $s-\epsilon<a$. \\ \\
    For the backward direction we assume that for all $\epsilon>0$ there exists $a\in A$ such that $s-\epsilon<a$. We know that $s$ is an upper bound for $A$ and, more importantly, we know from our assumption that any number less than $s$ cannot be an upper bound for $A$. Restated, if $b$ is an upper bound for $A$ then $s\leq b$. This gives us part (ii) of the supremum criteria. Therefore $s=\mathrm{sup}A$.
\end{proof}

\mytitlecompact{Section 1.3 - Exercises}

\begin{exer}{1.3.1}
    \begin{enumerate}
        \item[(a)]
            Write a formal definition in the style of Definition 1.3.2 for the \underline{infimum} or \underline{greatest lower bound} of a set. \\ \\
            A real number $i$ is the \underline{greatest lower bound} or \underline{infimum} for a set $A\in\R$ if it meets the following criteria:
            \begin{enumerate}
                \item[(i)] $i$ is a lower bound for $A$
                \item[(ii)] if $b$ is any lower bound for $A$, then $i\geq b$.
            \end{enumerate}
        \item[(b)]
            Now, state and prove a version of Lemma 1.3.8 for greatest lower bounds. \\ \\
            \emph{Assume $i\in\R$ is a lower bound for a set $A\subseteq\R$. Then, $i=\inf A$ if and only if, for every choice of $\epsilon>0$, there exists and element $a\in A$ satisfying $i+\epsilon>a$}.
            \begin{proof}
                For the forward direction we assume that $i=\inf A$ and we begin by observing that $i+\epsilon>i$ for all $\epsilon>0$. Since $i$ is the infimum for $A$ it must be the case that $i+\epsilon$ is not a lower bound. Therefore, there exists $a\in A$ for our arbitrary choice of $\epsilon$ satisfying $i+\epsilon>a$. \\ \\
                For the backward direction we assume that for all $\epsilon>0$ there exists an $a\in A$ such that $i+\epsilon> a$. We know that $i$ is an upper bound for $A$ and, more importantly, we know from our assumption that any number greater than $i$ cannot be a lower bound for $A$. Restated, if $b$ is a lower bound for $A$ then $i\geq b$. This gives us part (ii) of the infimum criteria. Therefore $i=\inf A$.
            \end{proof}
    \end{enumerate}
\end{exer}

\begin{exer}{1.3.2}
    Give an example of each of the following, or state that the request is impossible.
    \begin{enumerate}
        \item[(a)]
            A set $B$ with $\inf B\geq\sup B$. \\ \\
            Let $B={0}$. Then $\inf B=\sup B=0$.
        \item[(b)]
            A finite set that contains its infimum but not its supremum. \\ \\
            The request is impossible. The greatest element of any finite subset is the set's supremum.
        \item[(c)]
            A bounded subset of $\Q$ that contains its supremum but not its infimum. \\ \\
            Let $B=\{x\in\Q:0<x\leq 1\}$. Then $\inf B=0\notin B$ but $\sup B=1\in B$.
    \end{enumerate}
\end{exer}

\begin{exer}{1.3.3}
    \begin{enumerate}
        \item[(a)]
        Let $A$ be nonempty and bounded below, and define $B=\{b\in R:b$ is a lower bound for $A\}$. Show that $\sup B=\inf A$. \\ \\

    \end{enumerate}
\end{exer}


\end{document}
