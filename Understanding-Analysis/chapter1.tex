\documentclass[11pt]{article} 

\usepackage{graphicx}
\usepackage{amsmath,amssymb}
\usepackage{tikz}
\usepackage{textcomp}

\newcommand{\doctitle}{ }
\makeatletter
\renewcommand{\ps@headings}{%
\renewcommand{\@evenhead}{\parbox{\textwidth}{\hrulefill 
\fbox{\sffamily Understanding Analysis, Chapter 1 - Abbott }\hrulefill } }%
\renewcommand{\@oddhead}{\parbox{\textwidth}{\hrulefill 
\fbox{\sffamily Understanding Analysis, Chapter 1 - Abbott  }\hrulefill}}}
\makeatother


\linespread{.9}
\hoffset=-0in    \voffset=-.5in
\oddsidemargin=0in   \evensidemargin=0in
\topmargin=-.25in
\textwidth=6.5in   \textheight=9.5in
\columnseprule=.3pt  
\setlength{\parskip}{1em}



\usepackage{multirow}
\usepackage{multicol} 

\newcommand{\mytitle}[1]{{
\begin{centering}
{\Large \sffamily Understanding Analysis, Chapter 1 - Abbott}\\
\bigskip\bigskip{\Large \sffamily \bfseries{#1}}\\
\bigskip\bigskip\end{centering}
}}

\newcommand{\mytitlecompact}[1]{{

\hfill
{\Large \sffamily \bfseries{#1}}
\hfill
%\bigskip
}}


\newcommand{\mysection}[1]{{
\smallskip
\noindent
{\large \sffamily \bfseries{#1}}
}}


\newcommand{\mysubsection}[1]{{
\smallskip
{\sffamily \bfseries{#1}}
}}

\pagestyle{empty}
\pagestyle{headings}

\def\N{{\mathbb{N}}}
\def\Z{{\mathbb{Z}}}
\def\Q{{\mathbb{Q}}}
\def\R{{\mathbb{R}}}
\def\C{{\mathbb{C}}}

\usepackage{amsthm}
\newtheorem{innercustomthm}{Theorem}
\newenvironment{thm}[1]
    {\renewcommand\theinnercustomthm{#1}\innercustomthm}
    {\endinnercustomthm}
    
\newtheorem{innercustomdefn}{Definition}
\newenvironment{defn}[1]
    {\renewcommand\theinnercustomdefn{#1}\innercustomdefn\upshape}
    {\endinnercustomdefn}

\newtheorem{innercustomexmp}{Example}
\newenvironment{exmp}[1]
    {\renewcommand\theinnercustomexmp{#1}\innercustomexmp\upshape}
    {\endinnercustomexmp}

\newtheorem{innercustomexer}{Exercise}
\newenvironment{exer}[1]
    {\renewcommand\theinnercustomexer{#1}\innercustomexer\upshape}
    {\endinnercustomexer}

\begin{document}

\mytitlecompact{Section 1.1 - Results}

\begin{thm}{1.1.1}
    There is no rational number whose square is 2.
\end{thm}
\begin{proof}
    Assume for the sake of contradiction that there exists a rational number $x$ whose square is $2$. Since $x$ is rational, there must exist integers $p$ and $q$ such that $x=\frac{p}{q}$ with $q\neq0$ and $gcd(p,q)=1$. It follows that $\frac{p^2}{q^2}=2$ and $p^2=2q^2$. From this we can say that, because $p^2$ is even, $p$ must be even. That is, there exists $k\in\Z$ such that $p=2k$. Expressing $p$ in terms of $k$ gives $(2k)^2=2q^2$, and $2k^2=q^2$. From this we can say that, because $q^2$ is even, $q$ must be even. But this implies that both $p$ and $q$ are divisible by $2$, contradicting our assumption that $gcd(p,q)=1$. Therefore, our initial assumption must be false and it must be the case that there is no rational number whose square is $2$.
\end{proof}

\mytitlecompact{Section 1.2 - Results}

\begin{defn}{1.2.3} 
    Given two sets $A$ and $B$, a \underline{function} from $A$ to $B$ is a rule or mapping that takes each element $x\in A$ and associates with it a single element of $B$. In this case, we write $f:A\to B$. Given an element $x\in A$, the expression $f(x)$ is used to represent the element of $B$ associated with $x$ by $f$. The set $A$ is called the \underline{domain} of $f$. The \underline{range} of $f$ is not necessarily equal to $B$ but refers to the subset of $B$ given by $\{y\in B:y=f(x)$ for some $x\in A\}$.
\end{defn}

\begin{exmp}{1.2.5} \textbf{(Triangle Inequality)} The \emph{absolute value function} is so important that it merits the special notation $|x|$ in place of the usual $f(x)$ or $g(x)$. It is defined for every real number via the piecewise definition
\begin{center}
    $|x|=
    \begin{cases}
        x & \text{if}~x\geq 0 \\
        -x & \text{if}~x<0
    \end{cases}
    $
\end{center}
With respect to multiplication and division, the absolute value function satisfies
\begin{enumerate}
    \item[(i)] $|ab|=|a||b|$
    \item[(ii)] $|a+b|\leq |a|+|b|$ (Triangle Inequality)
\end{enumerate}
for all choices of $a$ and $b$.
\end{exmp}

\begin{thm}{1.2.6}
    Two real numbers $a$ and $b$ are equal if and only if for every real number $\epsilon>0$ it follows that $|a-b|<\epsilon$.
\end{thm}
\begin{proof}
    For the forward direction, let $a=b$. Then $\forall~\epsilon>0$ it is true that $\epsilon>|a-b|$ because $|a-b|=0$. For the backward direction, we will assume that $a\neq b$. Then $\exists~\epsilon_0=|a-b|>0$. But the existence of $\epsilon_0$ contradicts our premise that $|a-b|<\epsilon~\forall~\epsilon>0$. Therefore $a\neq$ does not imply $|a-b|<\epsilon~\forall~\epsilon>0$.
\end{proof}

\mytitlecompact{Section 1.2 - Exercises}

\begin{exer}{1.2.1} 
    (a) Prove that $\sqrt{3}$ is irrational. Does a similar argument work to show $\sqrt{6}$ is irrational? (b) Where does the proof of Theorem 1.1.1 break down if we try to use it to prove $\sqrt{4}$ is irrational?
    \begin{enumerate}
        \item[(a)] 
            \begin{proof}
                Assume for the sake of contradiction that there exists a rational number $x$ whose square is $3$. Since $x$ is rational, there must exist integers $p$ and $q$ such that $x=\frac{p}{q}$ with $q\neq 0$ and $gcd(p,q)=1$. It follows that $\frac{p^2}{q^2}=3$ and $p^2=3q^2$. From this we can say that, because $p^2$ is divisible by $3$, $p$ must be divisible by $3$. That is, there exists $k\in\Z$ such that $p=3k$. Expressing $p$ in terms of $k$ gives $(3k)^2=3q^2$, and $3k^2=q^2$. From this we can say that, because $q^2$ is divisible by $3$, $q$ must be divisible by $3$ But this implies that both $p$ and $q$ are divisible by $3$, contradicting our assumption that $gcd(p,q)=1$. Therefore, our initial assumption must be false and it must be the case that there is no rational number whose square is $3$.
            \end{proof} 
            A similar argument would work to show that $\sqrt{6}$ is irrational. You would arrive at the contradiction by showing that $p$ and $q$ are both divisible by $2$.
        \item[(b)]
            It is not necessarily true that if $n^2$ is divisible by $4$ then $n$ is divisible by $4$ (take $n=9$ for example).
    \end{enumerate}
\end{exer}
\begin{exer}{1.2.2}
    Show that there is no rational number $r$ satisfying $2^r=3$.
    \begin{proof}
        Assume for the sake of contradiction that there exists a rational number $r$ such that $2^r=3$. Since $x$ is rational, there must exist integers $p$ and $q$ such that $x=\frac{p}{q}$ with $q\neq 0$ and $gcd(p,q)=1$. It follows that $2^{\frac{p}{q}}=3$ and $2^p=3^q$. This equality cannot hold because $2^p$ is even and $3^q$ is odd. Therefore, there is no rational number $r$ satisfying $2^r=3$.
    \end{proof}
\end{exer}
\begin{exer}{1.2.3}
    Decide which of the following represent true statements about the nature of sets. For any that are false, provide a specific example where the statement in question does not hold.
    \begin{enumerate}
        \item[(a)] 
            If $A_1\supseteq A_2\supseteq A_3\supseteq A_4...$ are all sets containing an infinite number of elements, then the intersection $\cap_{n=1}^{\infty} A_n$ is infinite as well. \\ \\
            False. Consider the intersection $\cap_{n=1}^{\infty} A_n$ where $A_n=\{m\in\N~|~m\geq n\}$. The intersection must contain natural numbers but for every natural number $m$ there will always be a set $A_n$ that does not contain $m$, thus $m$ won't be in the intersection.
        \item[(b)] 
            If $A_1\supseteq A_2\supseteq A_3\supseteq A_4...$ are all finite, nonempty sets of real numbers, then the intersection $\cap_{n=1}^{\infty}A_n$ is finite and nonempty. \\ \\
            True. Each subset must be nonempty and finite so the intersection will always have at least one element and never be infinite.
        \item[(c)]
            $A\cap(B\cup C)=(A\cap B)\cup C$ \\ \\
            False. Choose $x$ such that $x\in C$ but $x\notin A$ and $x\notin B$. Then $x\in (A\cap B)\cup C$ but $x\notin A\cap(B\cup C)$.
        \item[(d)]
            $A\cap(B\cap C)=(A\cap B)\cap C$ \\ \\
            True. Any element in either set needs to be in $A$, $B$, and $C$.
        \item[(e)]
            $A\cap(B\cup C)=(A\cap B)\cup(A\cap C)$ \\ \\
            True. An element in either set must be in $A$ and either $B$ or $C$.
    \end{enumerate}
\end{exer}
\begin{exer}{1.2.4}
    Produce an infinite collection of sets $A_1,A_2,A_3,...$ with the property that every $A_i$ has an infinite number of elements, $A_i\cap A_j=\emptyset$ for all $i\neq j$ and $\cup_{i=1}^{\infty}A_i=\N$. \\ \\
    Construct the desired collection of sets by choosing $A_n=\{a_1, a_2, a_3,...\}$ where $a_1=\Sigma_{i=1}^{n}i$ and $a_m=a_{m-1}+(n+m-2)$ for all $m>1$.
\end{exer}
\begin{exer}{1.2.5} \textbf{(De Morgan's Laws)}
    Let $A$ and $B$ be subsets of $\R$.
    \begin{enumerate}
        \item[(a)] 
            If $x\in(A\cap B)^c$, explain why $x\in A^c\cup B^c$. This shows that $(A\cap B)^c\subseteq A^c\cup B^c$. \\ \\
            If $x\in(A\cap B)^c$ then $x$ is not in both $A$ and $B$. This means $x$ can be in one of either $A$ or $B$ or $x$ can be in neither. For the first case, without loss of generality, let $x\in A$ and $x\notin B$. Then $x\in B^c$ and thus $x\in A^c\cup B^c$. For the second case, $x\notin A$ and $x\notin B$. Then $x\in A^c$ and $x\in B^c$ so $x\in A^c\cup B^c$. Therefore, $x\in (A\cap B)^c\subseteq A^c\cup B^c$.
        \item[(b)]
            Prove the reverse inclusion $(A\cap B)^c\supseteq A^c\cup B^c$, and conclude that $(A\cap B)^c=A^c\cup B^c$. \\ \\
            If $x\in A^c\cup B^c$ then $x$ is not in $A$ or $x$ is not in $B$. Without loss of generality, let $x\notin A$. Then $x\notin A\cap B$ and thus $x\in(A\cap B)^c$. Therefore, $A^c\cup B^c\subseteq x\in (A\cap B)^c$. \\ \\
            Combining this result with part (a), we get $(A\cap B)^c=A^c\cup B^c$.
        \item[(c)]
            Show $(A\cup B)^c=A^c\cap B^c$ by demonstrating inclusion both ways. \\ \\
            If $x\in (A\cup B)^c$ then $x\notin A$ and $x\notin B$, thus $x\in A^c\cap B^c$. This gives us $(A\cup B)^c\subseteq A^c\cap B^c$. If $x\in A^c\cap B^c$ then $x\notin A$ and $x\notin B$, thus $x\notin(A\cup B)$ and $x\in(A\cup B)^c$. This gives us $A^c\cap B^c\subseteq(A\cup B)^c$. Therefore, $(A\cup B)^c=A^c\cap B^c$.
    \end{enumerate}
    
\end{exer}

\end{document}
